% Reiterate introduced problems
A central challenge in bridging the gap between neuro-scientific experimentation and theory is the modeling of neural dynamics. At the microscopic level of neurons, established nonlinear models effectively describe the essential functioning dynamics of the brain. Recent advances in neuroimaging techniques and larger collections of data hope to push the envelope of our understanding further in the macroscopic scale. Mathematical concepts are pushed to higher dimensions to accommodate the expansion in scale, tools in network theory are used to approximate relationships between pre-defined brain regions, and machine learning algorithms are used to infer parameters and correlations from data. However, is it safe to assume massive neuron networks on the whole-brain level to exhibit an even larger spectrum of nonlinear dynamics? Are nonlinear dynamical models intended for neurons adaptable to whole-brain recordings?

% Reiterate the findings from chapters, extrapolate on linear macroscopic dynamics


% Write about Dan, Kamalini, and Parul's future works and moving forward
\section{Going Forward}


% Importance of reproducbility, remarks on DWI chapter and SGM github
\section{Reproducibility \& Open Source}
