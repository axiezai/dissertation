% Reiterate introduced problems
A central challenge in bridging the gap between neuro-scientific experimentation and theory is the modeling of neural dynamics. At the microscopic level of neurons, established nonlinear models effectively describe the essential functioning dynamics of the brain. Recent advances in neuroimaging techniques and larger collections of data hope to push the envelope of our understanding further in the macroscopic scale. Mathematical concepts are pushed to higher dimensions to accommodate the expansion in scale, tools in network theory are used to approximate relationships between pre-defined brain regions, and machine learning algorithms are used to infer parameters and correlations from data. However, is it safe to assume massive neuron networks on the whole-brain level to exhibit an even larger spectrum of nonlinear dynamics? Are nonlinear dynamical models intended for neurons adaptable to whole-brain recordings?

% Reiterate the findings from chapters, extrapolate on linear macroscopic dynamics
Observed correlated brain activity is an indisputable evidence for operation and communication between interconnected brain regions. Therefore, Statistical estimates of correlated brain activity have become targets for potential biomarker investigations. Metrics such as connectivity, coherence, or synchrony are all second order statistics of mean brain activity, and such metrics are derived from sub-second recordings of brain activity in a brain volume with billions of firing neurons. While the lower dimensional representations are easier to quantify and digest for researchers, drawing causal conclusions and possible classifications based on these metrics are heavily contaminated by confounds. 

Despite heavily confounded metrics, functional connectivity matrices of brain activity are still often seen as a target for computational modeling. While recent computational network models have been ubiquitously used to decompose neural recordings underlying behavior and cognition, the field has faced a wall in advancing their findings to the  clinical setting due to the lack of generalizability and interpretation of their models. In Chapter \ref{chap:nmm}, we demonstrate that nonlinear mean field models meant for small populations of neurons are not directly translatable to macroscopic brain activity. In addition, we show that model inversion with connectivity metrics results in model parameters that do not reproduce biophysiological power spectrum, which are direct derivatives of recorded brain activity. 

Despite the importance of brain spectral content spatial distribution, recent efforts utilizing nonlinear mean field models do not examine the spatial properties of brain activity. In chapters \ref{chap:lap} and \ref{chap:sgm} we showcase our efforts in capturing the spatial and spectral properties of brain activity. By utilizing a more anatomically informative complex valued network, we were able to establish theory for a linear approach to modeling the structure and function relationship. Our models do not require time consuming numerical integration to solve, and our low dimensional parameters are easily interpretable. 

A recent study compared linear and nonlinear models of approximating macroscopic EEG and fMRI activity showed that linear models outperformed nonlinear models in terms of accuracy \cite{nozari_is_2020}, suggesting that linear approaches on the macroscopic scale have superior interpretability and performance despite inherent nonlinearities in the brain. Our framework showcased that there exists an organization of brain dynamics in the macroscopic scale that emerges as complex-valued eigenmodes. And linear models of brain dynamics utilizing combinations of these eigenmodes are rich in spatial information.

% Write about Dan, Kamalini, and Parul's future works and moving forward
% Importance of reproducbility, remarks on DWI chapter and SGM github
\section{Looking Forward}
My contributions provide a principled foundation for modeling brain dynamics with complex valued eigenmodes and the spectral graph model, and more exciting studies will be built upon the contents of this thesis. Dr. Parul Verma is experimenting with more details of the spectral graph model in hopes of further clarifying local neuron population contributions to macroscopic dynamics. Dr. Kamalini Ranasinghe, who has established group level differences in MEG power spectrum between healthy and patients suffering from neurodegeneration \cite{Ranasinghe2014}, will be using our spectral graph frameworks to further validate model driven approaches in the clinical setting. 

The most promising clinical application of our work is the effort by Dr. Danilo Bernardo. As a new faculty at UCSF focusing on pre-natal brain maturation, he found my online GitHub repository and took advantage of my reproducible code to produce preliminary results that led to a successful collaboration. His latest work will showcase developmental spectral maturation of brain activity is an age-dependent phenomenon, which can be captured by tuning of localized neural dynamics that plays out on the macroscopic brain networks, as modeled by the spectral graph model.

I hope our successful collaboration with Dr. Bernardo showcases the benefits of sharing reproducible code and data in scientific research. First and foremost, a publication that cannot be shared or reproduced by others is simply "dead upon publication". My published works are paired with online public GitHub repositories with working executable notebooks showcasing examples and analysis that can be ran by anyone with access to the internet. Such practices should become a minimal requirement for scientific publications, no contents should be locked behind proprietary software or publisher paywalls, as scientific research is funded by the public and should belong to the public. Organizations such as Brainhack, Neurohackademy, and ReproNim are pioneers in reproducibility and sharing, and I'd like to especially thank these welcoming communities for teaching me the valuable skills and essential practices that are translatable to any project.

