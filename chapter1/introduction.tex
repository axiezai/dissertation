%%% introduction
% plasticity vs evolution argument
Structural diversity and functional complexity of the human central nervous system is often placed into spotlight through the lens of evolution. While the sheer difference in cortex size might be the most striking difference between primate and other vertebrate brains, human brains stand out even more by its ability to invention and communicate symbol systems, which is poorly equipped in other primates. What special human brain functionality allowed us to invent cultural tools such as languages, numeric systems, and even arithmetics? As an exercise, one can consider two possible hypothesis at extreme ends of the structure-function argument:

\begin{itemize}
  \item Cultural functionality is acquired by expanded cortical plasticity unique to human brains, where newly acquired skills can be attributed to freshly established connections independent of an anatomical constraint.
  \item Humans have evolved specialized circuitry in the cortex, and specific brain circuitry contributes to unique cognitive functions. 
\end{itemize}

With the plastic brain scenario from  the first hypothesis, one can imagine if a group of people learned to play video games with novel control mechanisms, the plasticity induced changes in each person's brain may occur at various unpredictable locations in the brain. On the other hand, one would imply brain regions responsible for language and arithmetic are only found in humans. Of course, both of these extremes have been proven false by evidence. Specialists like musicians experience enhanced sensorimotor functions after repeated association, feedback, and cognitive practices, which are hallmarked by strengthened connections near the arcuate and intraparietal sulcus \cite{wan_music_2010}. Whereas mechanisms for vocal learning is identified in song birds \cite{warren_mechanisms_2011} and symbolic associations and computations in primate brains were also found to be associated with their prefrontal and parietal structures \cite{nieder_counting_2005,diester_semantic_2007}. If anatomical structures do support development of specialized functions in the brain, then how exactly does the complex connections made by billions of neurons give rise to the functional signals recorded by neuroscientists today? 

% Anatomy <--> function, link between neurobiology and observed phenomena 
Elucidating how structure shapes observable function is at the heart of a wide spectrum of scientific disciplines. Often, functional units have easily discernible structures governing their roles in a biological system. For example, the 3D molecular structure of a protein forming an ion channel receptor for transport across membranes. As the system becomes more complex, it becomes increasingly difficult to explain emergent function in relation to its underlying structure. Currently, the most complex physical system in the known universe is the central nervous system; the dense synaptic connections and staggering axonal projections in the brains of even simple organisms underly the myriad of fascinating behaviors in nature.

% structure function and why resting state
The relationship between the brain's structure and function is of particular interest in neuroscience, as morphological variants of the nervous system have repeatedly been shown to be associated with behavioral changes due to the brain's organizations \cite{sharp_default_2011, shen_using_2017}. 

A large body of work has been devoted to reproducing resting-state brain activity by means of computational modeling
%% Yeo's parcellation as example figure of clustering

% Data-driven and biophysically based modeling, 
%% Example of dynamical system states from Breakspear review
%% Rate models basics
Biophysical models of observable brain data are ... and parameter estimates of such models are a theoretically driven method to reduce the dimensionality of observable brain data for prediction of brain states and clinical statuses \cite{huys_computational_2016}.

Interpreting model parameter estimates require robust methods to evaluate their generative values, ideally including their uncertainty. 

% connectomics, DWI
% Example figure from nilearn?

% networks and graph theory in neuro modeling
Evidence of correlated activity is observed at the microscopic scale of communicating neurons, prompting extensive efforts to theoretically model these synchronous input  and output relationships (sources). While information from single neuron spike recordings can be sufficiently summarized with poisson distributions and probabilistic models (), high dimensional data from whole brain recordings with fMRI, EEG, and MEG require dimensionality reduction for meaningful interpretation. Graph theory and network theoretics have emerged as an advantagous tool in the field of neuroimaging. By defining specific brain regions of interest (ROIs) as nodes in a network, one can begin dissecting observed functional data 

% Paragraph about mappings, see Markello & Misic 2021 Neuroimage