%%% introduction
% Anatomy <--> function, link between neurobiology and observed phenomena 
Elucidating how structure shapes observable function is at the heart of a wide spectrum of scientific disciplines. Often, functional units have easily discernible structures governing their roles in a biological system. For example, the 3D molecular structure of a protein forming an ion channel receptor for transport across membranes. As the system becomes more complex, it becomes increasingly difficult to explain emergent function in relation to its underlying structure. Currently, the most complex physical system in the known universe is the central nervous system; the dense synaptic connections and staggering axonal projections in the brains of even simple organisms underly the myriad of fascinating behaviors in nature.

% structure function and why resting state
The relationship between the brain's structure and function is of particular interest in neuroscience, as morphological variants of the nervous system have repeatedly been shown to be associated with behavioral changes due to the brain's organizations \cite{sharp_default_2011, shen_using_2017}. 

A large body of work has beeen devoted to reproducing resting-state brain activity by means of computational modeling

% networks and graph theory in neuro modeling
Evidence of correlated activity is observed at the microscopic scale of communicating neurons, prompting extensive efforts to theoretically model these synchronous input  and output relationships (sources). While information from single neuron spike recordings can be sufficiently summarized with poisson distributions and probabilistic models (), high dimensional data from whole brain recordings with fMRI, EEG, and MEG require dimensionality reduction for meaningful interpretation. Graph theory and network theoretics have emerged as an advantagous tool in the field of neuroimaging. By defining specific brain regions of interest (ROIs) as nodes in a network, one can begin dissecting observed functional data 

% connectomics, DWI

% Data-driven and biophysically based modeling, 
Biophysical models of observable brain data are ... and parameter estimates of such models are a theoretically driven method to reduce the dimensionality of observable brain data for prediction of brain states and clinical statuses \cite{huys_computational_2016}.

Interpreting model parameter estimates require robust methods to evaluate their generative values, ideally including their uncertainty. 

% Paragraph about mappings, see Markello & Misic 2021 Neuroimage