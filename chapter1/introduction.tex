%%% introduction
% Anatomy <--> function, link between neurobiology and observed phenomena 
Elucidating how structure shapes observable function is at the heart of a wide spectrum of scientific disciplines. Often, functional units have easily discernible structures governing their roles in a biological system. For example, the 3D molecular structure of a protein forming an ion channel receptor for transport across membranes. As the system becomes more complex, it becomes increasingly difficult to explain emergent function in relation to its underlying structure. Currently, the most complex physical system in the known universe is the central nervous system; the dense synaptic connections and staggering axonal projcets in the brains of even simple organisms underly the myriad of fascinating behaviors in nature.

% paragraph on structure function stuff and why resting state
The relationship between the brain's structure and function is of particular interest in neuroscience, as morphological variants of the nervous system have repeatedly been shown to be associated with behavioral changes due to the brain's organizations \cite{sharp_default_2011,shen_using_2017}.


% networks and graph theory in neuro modeling

% connectomics, DWI

% Data-driven and biophysically based modeling, 