\section{Introduction}
Computational models in neuroscience has advanced to untangle observed neural signals at the macroscopic scale, with fMRI being the most common modality for model based approaches to decipher complex neural mechanisms \cite{wilson_is_2015}). As nonlinear behavior is inevitably encountered at the microscopic scale of individual neurons, neural mass models (NMMs) have emerged as a powerful approach to balance interpretability and biological relevance of computational models. Such models summarizes the state of locally interacting neuron populations with few parameters and a conversion from mean excitation level to mean population response \cite{freeman_tutorial_1992}. The conversion is typically performed via a nonlinear sigmoid function, whereas the mean firing rates, connection profiles, and membrane potentials are parameterized mathematically to model the lumped activity of particular brain regions\cite{LopesdaSilva1974, robinson_prediction_2001, Valdes1999}. The Wilson-Cowan single oscillator model \cite{Wilson1972} have evolved into a family of macroscopic NMMs in recent literature; with derivations for neocortical dynamics \cite{cowan_wilsoncowan_2016}, controllability of brain networks \cite{muldoon_stimulation-based_2016}, biomarkers in disease \cite{Zimmermann2018}, and second order statistics of observed brain signals \cite{Deco2009, abeysuriya_biophysical_2018, singh_estimation_2020, byrne_next-generation_2019, wang_inversion_2019}.

In a rare occurrence of public introspection amongst computational neuroscientists, Wilson and Niv's work \cite{wilson_is_2015} questioned whether model fitting is necessary for model-based analysis of fMRI. They addressed the weakness of models having free parameters, and the results of the analysis depend on free parameters are set. While their work was limited to the context of reinforcement learning and a single learning rate parameter, their conclusion is generalizable to the wider model-based neuroscience field: precise identification of parameters is not always necessary, but it is hard to identify neural correlates with model-based analysis due to sensitivity to parameters. More recently, Hartoyo et al. \cite{hartoyo_parameter_2019} disseminated the problem of unidentifiabiliy in whole brain models, where different parameters combinations can generate similar model predictions, especially in higher order multi-parameter dynamical systems. It has long been known that fitting of an unidentifiable model to data results in large uncertainties, out of the 22 unknown parameters from the linearized network model implemented by Hartoyo et al., only one parameter was found to be identifiable when fitted to EEG data. Nonetheless, the computational neuroscience field pushed the limit of neuron population level mean field models and extended them to describe neural activity at the whole brain scale.

% Unsolved problem and why is this a problem? 
Despite the prolific use of NMMs for whole brain model-based analysis, their nonlinear nature leads to difficulties in parameter inference and generalizable decoding of neural mechanisms. The saddle point and Hopf bifurcation behavior of the Wilson-Cowan model are neatly described by the original work of Wilson \& Cowan \cite{Wilson1972}, the Virtual Brain \cite{sanz-leon_mathematical_2015} and reviewed by Breakspear \cite{breakspear_dynamic_2017}. Such nonlinear systems are silent or in steady-state for most parameter regimes, but interesting oscillating behavior is observed when network coupling or external driving force parameters push the system over the Hopf bifurcation. During parameter inference, constant switching between regimes and model behavior leads to many local minimas during parameter optimization. To avoid this non-convex problem and parameter identifiability issues, a common practice is to sets all biophysiological parameters for neural subpopulations in a NMM to be near an appropriate Hopf bifurcation point, then one or few global parameters are optimized to second order functional connectivity (FC) metrics such as brain regions pairwise correlation or synchrony \cite{Zimmermann2018, Deco2009, abeysuriya_biophysical_2018, wang_inversion_2019, demirtas_hierarchical_2019, honey_predicting_2009}. However, this leaves a few problems; How does model optimization perform when inferring more than one global parameter at a time? Does the fMRI FC metric fitting translate well to the encephalography modalities? And finally whether the model parameters optimized for FC metrics translate well to the frequency spectrum?

% Variations of the network Wilson-Cowan model has been shown to decode neural signals by replicating functional connectivity metrics such as BOLD fMRI correlation, coupling, and phase lock metrics \cite{abeysuriya_biophysical_2018}. However, the set of parameters estimating second order statistical maps may not reproduce more direct neural derivatives such as the power spectrum density. Additionally, 

The traditional practice of only finding one or two optimal global connectivity related parameters, sometimes by manual grid search(see all works in Table \ref{tab:nmm_pubs}), while having its merits, is a fault of the high dimensional and nonlinear nature of NMMs. Techniques such as stochastic gradient descent found success in deep learning mainly due to the overparameterization of networks and the theory behind the universal approximation theorem \cite{lu_expressive_2017, zhou_universality_2020}. On the other hand, biophysiological models do not have the luxury of overparameterization, making any gradient descent algorithms difficult to implement for model optimization. In this work, we will investigate and discuss the difficulties faced in gradient descent based approaches to parameter inference, as well as address the short-comings of current network NMMs. We present a systematic examination of the Wilson-Cowan model. First, we examine the performance of a single Wilson-Cowan oscillator unit when fitting to broad band spectra, and whether MCMC sampling can converge to a reasonable posterior distribution for its parameters. In addition, we do not expect a single oscillator model to be able to capture the properties of an entire MEG frequency spectrum. However, if introducing coupling and delays to identical brain regions creates more dynamical oscillations as suggested by (\cite{Deco2009}), then a systematic examination of the new objective function space should be made. We implement the whole brain network Wilson-Cowan model to sample for posteriors that captures whole brain MEG derivatives such as coherence functional connectivity (COH) and amplitude envelope correlation (AEC), with the goal to verify whether the global parameters themselves can capture these metrics well, and whether the inferred parameters for FC produces a biophysiological power spectrum.

\begin{table}
 \caption{Whole Brain Neural Mass Model Parameter Inference Publications and their performance. This table does not include publications with whole brain mean field models or mechanistic models of neural activity. }
  \centering
  \begin{tabular}{llll}
    \toprule
    \cmidrule(r){1-2}
    Name & Modality  & Target  & Highest Accuracy \\
    \midrule
    Zimmerman et al. \cite{Zimmermann2018} & fMRI & Correlation FC & $r=0.57$  \\
    Honey et al. \cite{honey_predicting_2009} & fMRI & Correlation FC & $ r=0.48$ \\
    Demirtas et al. \cite{demirtas_hierarchical_2019} & fMRI & Correlation FC & $r=0.743$ \\
    Wang et al. \cite{wang_inversion_2019} & fMRI & Correlation FC & $r=0.46$ \\
    Schirner et al. \cite{schirner_inferring_2018} & fMRI & BOLD Time series & $r=0.50$ \\
    Falcon et al. \cite{falcon_virtual_2015} & fMRI  & Correlation FC  &  $r = 0.29$  \\
    Abeysuriya et al. \cite{abeysuriya_biophysical_2018} & MEG & Synchrony FC & $r = 0.48$ \\
    Deco et al. (2017) \cite{deco_single_2017}  & MEG & Envelope FC & $r=0.45$ \\
    Deco et al. (2009) \cite{Deco2009} & fMRI & Kuramoto order parameter & Not reported \\
    Hadida et al. \cite{hadida_bayesian_2018} & MEG & Envelope FC & $r=0.42$ \\
    \bottomrule
  \end{tabular}
  \label{tab:nmm_pubs}
\end{table}