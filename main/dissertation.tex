\documentclass[phd,tocprelim]{cornell}
%
% tocprelim option must be included to put the roman numeral pages in the
% table of contents
%
% The cornellheadings option will make headings completely consistent with
% guidelines.
%
% This template was originally provided by Blake Jacquot, and
% fixed up by Andrew Myers.
%
%Some possible packages to include
\usepackage[utf8]{inputenc}
\usepackage{graphicx,pstricks}
\usepackage{microtype}
\usepackage{graphics}
\usepackage{moreverb}
\usepackage{subfigure}
\usepackage{epsfig}
\usepackage{subfigure}
\usepackage{hangcaption}
\usepackage{txfonts}
\usepackage{palatino}
% Packages added by Xihe:
\usepackage{subfiles}
\usepackage{bm}
\usepackage[font=small, labelfont=bf]{caption}
\usepackage{array} % for flexible tables
\usepackage{hyperref} % for links
\usepackage{amsbsy} % for bold
\usepackage{caption}
\usepackage{indentfirst} % indent first paragraph after \section tags

%if you're having problems with overfull boxes, you may need to increase
%the tolerance to 9999
\tolerance=9999

\bibliographystyle{unsrt}
%\bibliographystyle{IEEEbib}

\renewcommand{\caption}[1]{\singlespacing\hangcaption{#1}\normalspacing}
\renewcommand{\topfraction}{0.85}
\renewcommand{\textfraction}{0.1}
\renewcommand{\floatpagefraction}{0.75}

\title{Model Based Analysis of Multimodal Neuroimaging: From Neural Masses to Spectral Graph Theory}
\author {Xihe Xie}
\conferraldate {November}{2021}
\degreefield {Ph.D.}
\copyrightholder{Xihe Xie}
\copyrightyear{2021}

\begin{document}

\maketitle
\makecopyright

\begin{abstract}
\subfile{../abstract/abstract}
\end{abstract}

\begin{biosketch}
Xihe Xie was born in Beijing, China in 1991. After completing elementary school in China, Xihe moved to the United States to reunite with his family and resumed his schooling in Forest Hills, New York. While attending Francis Lewis High School, he encountered charismatic teachers in his STEM classes, sparking his initial interests in science and engineering. Before moving onto The City College of New York with a full scholarship, Xihe left his mark at Francis Lewis High School by organizing students and teachers to fund raise for a robotics team, which went on to compete in the FIRST robotics competitions and eventually grew into a specialized STEM curriculum at the high school. In college, Xihe played on the Men's volleyball team and became team captain in his senior year. During the off-season, Xihe joined the labs of Dr. Lucas Parra and Dr. Marom Bikson, where his fascination with high dimensional data and signal processing concepts was put to use in neuroscience for the first time. Xihe completed his undergraduate degree in Biomedical Engineering as \emph{magna cum laude} and a Tau Beta Pi Engineering Honor society inductee. 

During the two years after graduation, Xihe took on two junior level biomedical device jobs in industry, but quickly grew weary of the entry level tasks after completing training. But now having some expendable income, Xihe would quit both of his jobs during the summers and backpacked through 11 countries in Asia and Europe. Upon Xihe's return from his second trip, he was eager to begin pursuing his neuroscience Ph.D. and resume learning in a more curious academic environment. Xihe began his graduate school career as an early enrollee in July 2015 at Weill Cornell Graduate School of Biomedical Sciences in the lab of Dr. Keith Purpura. Soon after, Xihe would join the lab of Dr. Ashish Raj to investigate multi-modal neuroimaging and models of the human brain's structure-function relationship. Despite an unexpected move to UCSF led by Dr. Raj, Xihe published two research articles through collaborations with Dr. Srikantan Nagarajan at UCSF, two commentary articles as part of the Brainhack community, an open source software, numerous collaborations, and a manuscript for future publication at the time of graduation. Xihe took on many volunteering, teaching assistant, and organizer roles for Brainhack and Neuromatch events and was awarded the Repronim/INCF Fellowship in 2020. 

Beyond the lab, Xihe was actively involved in both the Weill Cornell and UCSF communities. He was the sports and gym liason for the student government at Weill Cornell, and he became an officer at UCSF's Open Science Group in his short time on the West Coast. In addition to volunteering at educational outreach programs in NYC, he was a teaching assistant and workshop speaker at many hackathons. As part of his ReproNim/INCF Fellowship training, Xihe created a course with Dr. Amy Kuceyeski named \emph{data science basics in neuroscience} to fulfill the graduate school's quantitative curriculum requirement, but with a more practical and modern approach. With his newly acquired hackathon skills, Xihe contributed to open source projects such as Jupyter and Nipy. Xihe's most treasured memories away from his work laptop are the scenic road trips with his graduate school friends and the sunny days spent playing beach volleyball at Ocean Beach or Brooklyn Bridge Park.
\end{biosketch}

\begin{dedication}
\emph{Dedicated to Guang Xie}
\end{dedication}

\begin{acknowledgements}
First and foremost, I would like to thank Dr. Ashish Raj and Dr. Srikantan Nagarajan, two advisors that provided opportunities and countless valuable advice from Weill Cornell to UCSF. Ashish always pushed the boundaries and was open to new things, Sri asked important questions and taught me how to critically approach the problems and tasks at hand. I am very appreciative of my thesis committee: Dr. Amy Kuceyeski, Dr. Keith Purpura, and Dr. Jonathan Victor, for sharing their expertise which was invaluable at every meeting. Thank you as well to Dr. Pablo F. Damasceno for accompanying me on pursuits of better data practices and mentorship advice.

I am extremely grateful for the Neuroscience Department at Weill Cornell Graduate School for providing a welcoming community and outstanding opportunities for teaching and peer led learning. Thanks to UCSF's Open Science Group, the Brainhack organizers, and Neurohackademy instructors, communities that kept me energized and shared valuable knowledge in scientific knowledge, publishing, ethics, and data practices. Thank you to the ReproNim team, INCF, and the ReproNim Fellowship trainers for broadening my horizons on the possibilities of scientific efforts. 

Thank you to my fellow scientists and friends at Weill Cornell for their expertise and support that got me through first-year courses. Thank you to my group of classmates and friends: Thomas, Rosa, Mitchell, and George who always showered me with warm welcomes whenever I visited from SF. Last, but not least, thank you to my Family. My parents have been incredibly supportive throughout my entire life, and I am lucky to have caring extended families in New York City, Beijing, and Chiba. To my parents, thank you for being there for me despite your insanely busy work schedule, and thank you for always thinking of me whenever food is involved.  
%Finally, many scientists and friends have given selflessly their expertise and support, including: 
\end{acknowledgements}

\contentspage
\tablelistpage
\figurelistpage

\normalspacing \setcounter{page}{1} \pagenumbering{arabic}
\pagestyle{cornell} \addtolength{\parskip}{0.5\baselineskip}

\chapter{Introduction}
\subfile{../chapter1/introduction}

\chapter{Advances in Diffusion MRI}
\subfile{../chapter2/dwi}

\chapter{Whole Brain Network Neural Mass Models: A Criticism}

\chapter{Emergence of Canonical Functional Networks from the Structural Connectome}
\subfile{../chapter4/chapter4_lap}

\chapter{Spectral Graph Theory of Brain Oscillations}
\subfile{../chapter5/chapter5_sgm}


\appendix
\chapter{Appendix: Supplementary Figures}
\subfile{../appendix/supplements}

\bibliography{References}

\end{document}
