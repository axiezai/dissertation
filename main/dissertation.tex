\documentclass[phd,tocprelim]{cornell}
%
% tocprelim option must be included to put the roman numeral pages in the
% table of contents
%
% The cornellheadings option will make headings completely consistent with
% guidelines.
%
% This sample document was originally provided by Blake Jacquot, and
% fixed up by Andrew Myers.
%
%Some possible packages to include
\usepackage[utf8]{inputenc}
\usepackage{graphicx,pstricks}
\usepackage{microtype}
\usepackage{graphics}
\usepackage{moreverb}
\usepackage{subfigure}
\usepackage{epsfig}
\usepackage{subfigure}
\usepackage{hangcaption}
\usepackage{txfonts}
\usepackage{palatino}
\usepackage{subfiles}
\usepackage{bm}
\usepackage[font=small, labelfont=bf]{caption}
\usepackage{array} % for flexible tables
%if you're having problems with overfull boxes, you may need to increase
%the tolerance to 9999
\tolerance=9999

\bibliographystyle{unsrt}
%\bibliographystyle{IEEEbib}

\renewcommand{\caption}[1]{\singlespacing\hangcaption{#1}\normalspacing}
\renewcommand{\topfraction}{0.85}
\renewcommand{\textfraction}{0.1}
\renewcommand{\floatpagefraction}{0.75}

\title {Emergence of neuronal dynamics from brain structure in multi-modal resting-state brain imaging}
\author {Xihe Xie}
\conferraldate {November}{2021}
\degreefield {Ph.D.}
\copyrightholder{Xihe Xie}
\copyrightyear{2021}

\begin{document}

\maketitle
\makecopyright

\begin{abstract}
\subfile{../abstract/abstract}
\end{abstract}

%\begin{biosketch}
%Your biosketch goes here. Make sure it sits inside
%the brackets.
%\end{biosketch}

\begin{dedication}
This document is dedicated to all Cornell graduate students.
\end{dedication}

\begin{acknowledgements}
The author wishes to acknowledge the generous contribution of several individuals and organizations that have made this dissertation posssible. Foremost is Dr. Ashish Raj and Dr. Srikantan Nagarajan, two advisors that provided opportunities and countless valuable advice from Weill Cornell to UCSF. 

The Neuroscience Department at Weill Cornell Graduate School for providing a welcoming community and outstanding opportunities for teaching and peer led leaerning. UCSF's Open Science Group, the Brainhack community, and Neurohackademy kept me energized and shared valuable knowledge in scientific knowledge, publishing, ethics, and data practices. The ReproNim team, INCF, and the ReproNim Fellowship trainers and trainees. Finally, many scientists and friends have given selflessly their expertise and support, including: 
\end{acknowledgements}

\contentspage
\tablelistpage
\figurelistpage

\normalspacing \setcounter{page}{1} \pagenumbering{arabic}
\pagestyle{cornell} \addtolength{\parskip}{0.5\baselineskip}

\chapter{Introduction}
\subfile{../chapter1/introduction}


\chapter{Spectral Graph Theory of Brain Oscillations}
\subfile{../chapter5/chapter5_sgm}
% \section{SECTION 1}
% The text for Section 1 goes here, without brackets.

% \section{SECTION 2}
% Section 2 text.

% \subsection{Subsection heading goes here}

% Subsection 1 text

% \subsubsection{Subsubsection 1 heading goes here}
% Subsubsection 1 text

% \subsubsection{Subsubsection 2 heading goes here}
% Subsubsection 2 text

% \section{SECTION 3}
% Section 3 text. The dielectric constant at the air-metal interface
% determines the resonance shift as absorption or capture occurs.

% \begin{equation}
% k_1=\frac{\omega }{c({1/\varepsilon_m + 1/\varepsilon_i})^{1/2}}=k_2=\frac{\omega
% sin(\theta)\varepsilon_{air}^{1/2}}{c}
% \end{equation}

% \noindent
% where $\omega$ is the frequency of the plasmon, $c$ is the speed of
% light, $\varepsilon_m$ is the dielectric constant of the metal,
% $\varepsilon_i$ is the dielectric constant of neighboring insulator,
% and $\varepsilon_{air}$ is the dielectric constant of air.

\chapter{Chapter 3}

\chapter{Chapter 4}

\appendix
\chapter{Chapter 1 of appendix}
Appendix chapter 1 text goes here

\bibliography{References}

\end{document}
