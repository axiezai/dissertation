The direct link between human neurobiology and observed brain dynamics drives fundamental research efforts in neurosience. How does functional brain patterns arise from the underlying anatomy? Alongside technological advances, multimodal brain imaging enlarged the coverage of observable brain characteristics, and data-driven network theoretics emerged as a valuable framework for understanding large datasets and building biophysically based models of brain structure and function. This dissertation explores structure function models of different complexity and spatiotemporal scale, utilizing tools from signal processing, dynamical systems, optimization, and stochastic processes. 

With diffusion imaging derived white matter streamlines, we built whole brain networks describing the underlying anatomical connections of the brain. The combination of connection strengths and inter-region delays provided anatomical networks that were rich in information. We showcase linear, low dimensional and highly interpretable network models of brain function that fully utilizes the brain's anatomical connectivity. Moreover, we found these models to capture the spatial patterns of brain activity in addition to observed functional patterns of the brain. We also examine a non-linear neural mass model of neuronal mean-field oscillations, in order to determine how parameterization and model complexity dictate model performance.
