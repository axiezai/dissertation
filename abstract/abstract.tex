The direct link between human neurobiology and observed brain dynamics drives fundamental research efforts in neurosience. Alongside technological advances, multimodal brain imaging enlarged the coverage of observable brain characteristics, and data-driven network theoretics emerged as a valuable framework for understanding high dimensional datasets and building biophysical models of brain function. This dissertation explores brain patterns arising from the underlying anatomy with structure-function models of different complexity and spatio-temporal scale, pairing theoretical models with tools from dynamical systems, signal processing, and optimization. 

Non-invasive brain imaging has numerous common sources of variability, here we showcase workflows featuring increased precision and accelerated computing speed. Through these automated workflows, networks describing the underlying white-matter connections of the brain were obtained as the structural basis of our work. The combination of connection strengths and inter-region delays provided anatomical networks that were rich in information. We find in nonlinear neural mass models summarizing neuron population firing rate, sub-optimal inference and practical shortcomings hinder analysis when extending to the whole brain network. We then showcase linear, low dimensional network models of brain function that utilize the brain's anatomical connectivity, reproducing spatial patterns in the slower blood-oxygen-level-dependent (BOLD) regimes as well as highly oscillatory encephalography frequencies. With ideas originating from spectral graph theory, our interpretable parameters capture the spatial distributions of brain activity in addition to observed functional patterns, supporting recent findings that suggest the resting brain is macroscopically linear.